%%%%%%%%%%%%%%%%%%%%%%%%%%%%%%%%%%%%%%%%%
% "ModernCV" CV and Cover Letter
% LaTeX Template
% Version 1.1 (9/12/12)
%
% This template has been downloaded from:
% http://www.LaTeXTemplates.com
%
% Original author:
% Xavier Danaux (xdanaux@gmail.com)
%
% License:
% CC BY-NC-SA 3.0 (http://creativecommons.org/licenses/by-nc-sa/3.0/)
%
% Important note:
% This template requires the moderncv.cls and .sty files to be in the same 
% directory as this .tex file. These files provide the resume style and themes 
% used for structuring the document.
%
%%%%%%%%%%%%%%%%%%%%%%%%%%%%%%%%%%%%%%%%%

\makeatletter
\@ifpackageloaded{moderncvstyleclassic}{%
\let\oldsection\section%
\renewcommand{\section}[1]{\leavevmode\unskip\vspace*{-\baselineskip}\oldsection{#1}}%
}{%
}
\makeatother

%----------------------------------------------------------------------------------------
%	PACKAGES AND OTHER DOCUMENT CONFIGURATIONS
%----------------------------------------------------------------------------------------

\documentclass[12pt,a4paper,sans]{moderncv} % Font sizes: 10, 11, or 12; paper sizes: a4paper, letterpaper, a5paper, legalpaper, executivepaper or landscape; font families: sans or roman

\moderncvstyle{classic} % CV theme - options include: 'casual' (default), 'classic', 'oldstyle' and 'banking'
\moderncvcolor{blue} % CV color - options include: 'blue' (default), 'orange', 'green', 'red', 'purple', 'grey' and 'black'

\usepackage{savesym}
\savesymbol{fax}

\usepackage{marvosym}
\usepackage{fontawesome}
\usepackage{lipsum} % Used for inserting dummy 'Lorem ipsum' text into the template
\usepackage[utf8]{inputenc}
\usepackage[scale=0.75]{geometry} % Reduce document margins
%\setlength{\hintscolumnwidth}{3cm} % Uncomment to change the width of the dates column
%\setlength{\makecvtitlenamewidth}{10cm} % For the 'classic' style, uncomment to adjust the width of the space allocated to your name

%----------------------------------------------------------------------------------------
%	NAME AND CONTACT INFORMATION SECTION
%----------------------------------------------------------------------------------------

\firstname{Gennaro} % Your first name
\familyname{Raiola} % Your last name

% All information in this block is optional, comment out any lines you don't need
\title{Curriculum Vitae}
%\address{Via Ariosto 3}{16159 Genova, IT}
\email{gennaro.raiola@gmail.com}
\homepage{github.com/graiola}
%{web} % The first argument is the url for the clickable link, the second argument is the url displayed in the template - this allows special characters to be displayed such as the tilde in this example
%\extrainfo{additional information}
%\photo[100pt][0.4pt]{photo_sea.png} % The first bracket is the picture height, the second is the thickness of the frame around the picture (0pt for no frame)
%\quote{}

%----------------------------------------------------------------------------------------

\begin{document}

\makecvtitle % Print the CV title


%\begin{center}
%\textit{Hard-working, flexible Robotics Engineer with a strong profession in clean code development; seeking new challenges to foster autonomous devices with powerful hardware and artificial intelligence; Passionate and goal oriented team player \\
%\#Robotics, \#Control, \#Machine Learning, \#Software Engineering, \#ROS, \#C++, \#Python, \#Matlab,  
%}
%\end{center}


\section{Fields of interest}
Robotics, programming, human-robot interaction, motion control, force control, inverse kinematics and dynamics, whole-body control, hardware integration, embedded, DevOps, ROS contributor.

\section{Profession}

\cventry{09/2019--10/2020}{Post-Doc}{Jet Propulsion Laboratory (NASA-JPL)  \href{https://www.jpl.nasa.gov/}{\Mundus}}{Pasadena, CA}{United States}{
At JPL I worked on demonstration of autonomous berthing, assembly and installation of scientific payloads using a robotic platform mounted on a testbed simulating a scientific space station. 
\begin{itemize}
\item Autonomous berthing:
\begin{itemize}
\item created a state estimator to track the full pose and velocity of an incoming payload using Kalman filtering via a fiducial detector,
\item developed a new software module for the testbed to generate coordinated motions of the robotic arm in order to intercept and dock the incoming payload by means of inverse kinematics and force control,
\item performed experiments in preparation for the paper \emph{Validating an Architecture for Robotic Assembly and Servicing of Hosted Payloads on a Persistent Platform},
\item gained experience with the JPL testbed software architecture and the m3tk simulation software,
\item integrated some of the functionalities of the testbed with the robotic operating system (ROS).
\end{itemize}
\item Autonomous assembly:
\begin{itemize}
\item developed code to perform assembly of instruments using fiducial movements for localization and positioning of the instrument and force control for interaction,
\item performed two demos in which the robot is able to autonomously assemble a starshade and a satellite dish,
\item created new modules and autonomous behaviors for the testbed.
\end{itemize}
\end{itemize}}

\cventry{09/2017--06/2019}{Post-Doc}{Istituto Italiano di Tecnologia (IIT)  \href{https://www.iit.it/research/lines/dynamic-legged-systems}{\Mundus}}{Genoa}{Italy}{
During my work at IIT, I had the opportunity to work on several aspects of the development and maintenance of software and electronics for the quadruped robots HyQ and HyQReal \href{https://www.youtube.com/watch?v=pLsNs1ZS_TI&t=1s}{\faPlayCircle}.
\begin{itemize}
\item Developed a real-time control framework and communication system with EtherCAT,
\item developed the software control architecture with ROS-Control, in order to make the robot capable of executing different types of gaits (e.g. crawl, trot, etc.) in complex and changing terrains,
\item developed the low level safety software layer to protect the robot hardware and human operators.
\item DevOps processes, including: 
\begin{itemize}
\item creating and maintaining fully automated CI/CD pipelines for code testing and deployment using GitLab-CI,
\item deploying apt servers to track the software dependencies,
\item developing Docker containers for code testing and development. 
\end{itemize}
\item Sensor integration and calibration for the HyQReal robot.
\item Research work to create a novel whole-body locomotion framework for quadrupedal robots using inverse dynamics and task optimization which led to the publication of the journal paper \emph{A simple yet effective whole-body locomotion framework for quadruped robots} \href{https://github.com/graiola/wbc-setup}{\faGithub}.
\item Collaboration with external work groups such as Moog and Vodafone to define the requirements and functionalities of the robots for various real-world scenarios.
\end{itemize}}

\cventry{03/2017--07/2017}{Post-Doc}{Robotics and Mechatronics group, University of Twente \href{https://www.ram.ewi.utwente.nl/}{\Mundus}}{Enschede}{The Netherlands}{
At the University of Twente, I worked on the development of a safety- and energy-aware impedance controller for the KUKA LWR 4+ robotic arm \href{https://www.youtube.com/watch?v=tFSSHXn5cGI}{\faPlayCircle}. 
\begin{itemize}
\item publication of the journal paper \emph{Development of a Safety and Energy Aware Impedance Controller for Collaborative Robots} on IEEE Robotics and Automation Letters which was selected for presentation at ICRA 2018.
\end{itemize}}

\cventry{01/2016-12/2016}{Ph.D. student in Robotics}{CEA-List (French Alternative Energies and Atomic Energy Commission - Laboratory for Integration of Systems and Technology) \href{http://www-list.cea.fr/en/}{\Mundus}}{Gif-sur-Yvette}{France}{
During the last year of my PhD, I had the opportunity to transfer my research results to a startup at CEA-List (\href{https://www.isybot.com/}{\underline{ISybot}}). 
\begin{itemize}
\item development of a force controller to generate virtual guides through kinesthetic teaching to be used within the software framework of the startup's collaborative robot \href{https://www.youtube.com/watch?v=dYPGD8K_-hc}{\faPlayCircle}, \href{https://github.com/graiola/virtual-fixtures}{\faGithub}.
\end{itemize}}

\cventry{05/2013--12/2013}{Research Engineer in Motion Control of Humanoid Robots}{PAL Robotics S.L. \href{http://www.pal-robotics.com}{\Mundus}}{Barcelona}{Spain}{
At PAL Robotics, I worked on a team to design and test the ROS-Control package. The aim of ROS-Control is to implement a Hardware Abstraction Layer for different kinds of robots (e.g. humanoids, manipulators, mobile robots, etc.) \href{https://github.com/ros-controls}{\faGithub}. 
\begin{itemize}
\item implemented via ROS-Control an inverse kinematics controller with task optimization for REEM-H and REEM-C robots in collaboration with LAAS-CNRS in France \href{https://www.youtube.com/watch?v=lmE5RmvkAhQ}{\faPlayCircle}.
\end{itemize}}

\cventry{09/2012--02/2013}{Intern}{ENSTA-ParisTech and UPMC-ISIR \href{http://www.isir.upmc.fr/}{\Mundus}}{Paris}{France}{
I developed a library in Matlab and C++ to generate motion primitives and perform skill optimization for humanoid robots (MEKA, NAO, ICub and Pepper) \href{https://github.com/stulp/dmpbbo}{\faGithub}.
This library has been successfully used with the SoftBank robot Pepper to learn and play the "ball in the cup" dexterous game \href{https://www.youtube.com/watch?v=jkaRO8J_1XI}{\faPlayCircle}.}

%----------------------------------------------------------------------------------------
%	EDUCATION SECTION
%----------------------------------------------------------------------------------------
\section{Education}

\cventry{2014--2016}{Ph.D. student in Robotics}{Université Paris-Saclay \href{https://www.universite-paris-saclay.fr/fr}{\Mundus}}{Palaiseau}{France}{}
\cventry{2009--2012}{Master's Degree (M.Sc) cum laude in Automation and Control Engineering}{University of Naples "Federico II"}{Naples}{Italy}{}
\cventry{2006--2009}{Bachelor's Degree (B.Sc) in Computer Engineering}{University of Naples "Federico II"}{Naples}{Italy}{}

%\vspace{0.25cm}

%\section{Research Interests}

%Controls, Software Engineering, Machine Learning, Motion Primitives, Collaborative robotics. 

%\section{Research}
%\cvitem{Post-doc}{Energy based control strategies for collaborative robots \newline{}
%	\footnotesize{
%	\underline{Keywords}: Safety in robotics, Impedance Control, Collaborative robotics.}}

%\cvitem{Ph.D. Thesis}{"Co-manipulation with a library of Virtual Guides" \newline{}
%	\footnotesize{\underline{Supervisors}: Prof. Freek Stulp (ENSTA-ParisTech), Dr. Xavier Lamy (CEA-List). \newline{}
%	\underline{Keywords}: Collaborative robotics, Industrial applications, Controls, Machine Learning.}}

%\cvitem{M.Sc. Thesis}{"Learning Parameterized Skills through Models with Expanded Kernels" \newline{} 
%	\footnotesize{\underline{Supervisors}: Prof. Bruno Siciliano (PRISMA lab.) and Prof. Freek Stulp (ENSTA-ParisTech). \newline{}
%	\underline{Keywords}: Motion Primitives, Machine Learning.}}

%\vspace{2.0cm}

%\hspace{2em}

%\section{Technical Skills and Software projects}

%\subsection{Technical Skills}
%\cvitem{Programming}{C/C++, Python, MATLAB, ROS, Orocos, Qt, Boost, RTAI/Xenomai, Eigen, php}
%\cvitem{Systems/Tools}{Linux/Ubuntu, CMake, Latex, QtCreator, Gazebo, Eclipse, git, VirtualBox, Docker, EtherCAT}
%\cvitem{Hardware}{Arduino}

%\subsection{Software projects}

\section{Technical skills}

\cvitem{\listitemsymbol}{Proficient in the following programming languages: \textbf{C}, \textbf{C++} and \textbf{Matlab}}
\cvitem{\listitemsymbol}{Competent with \textbf{Python} and \textbf{Bash} scripting.}
\cvitem{\listitemsymbol}{Competent with \textbf{Qt}, \textbf{Eigen} and \textbf{ROS}}
\cvitem{\listitemsymbol}{Excellent knowledge of \textbf{GIT}.}
\cvitem{\listitemsymbol}{Excellent knowledge of \textbf{CMake} and \textbf{Makefile} for managing the build process of software and Doxygen for code documentation.}
\cvitem{\listitemsymbol}{Competent with \textbf{Docker} and \textbf{Virtual Machines} deployment for testing and development.}
\cvitem{\listitemsymbol}{Deep knowledge of \textbf{Linux}-based operating systems (Ubuntu, Kali, Debian).}
\cvitem{\listitemsymbol}{Experienced with real time operating systems \textbf{RTOS} (RTAI, Xenomai, RT-PREEMPT), Kernel configuration and \textbf{EtherCAT}.}

%\section{Selected open-source software projects}

%\cvitem{\listitemsymbol}{"wbc - Whole Body Control for quadruped robots" 
%\href{https://github.com/graiola/wbc-setup}{\faGithub \underline{}}
%\href{https://www.youtube.com/watch?v=omQtma3-R94}{\faPlayCircle \underline{}}
%\newline{}
%\footnotesize{Demo using the whole body controller I developed for HyQ running inside a docker container.}}

%\cvitem{\listitemsymbol}{"ros-control" 
%\href{https://github.com/ros-controls}{\faGithub \underline{}}
%\newline{}
%\footnotesize{Ros packages to make controllers generic to all robots.}}

%\cvitem{\listitemsymbol}{"Stack-of-Task" 
%\href{http://stack-of-tasks.github.io/authors.html}{\faGithub \underline{}}
%\href{https://www.youtube.com/watch?v=lmE5RmvkAhQ}{\faPlayCircle \underline{}} 
%\newline{}
%\footnotesize{Integration of the stack of tasks inverse kinematics solver on the REEM-H and REEM-C robots at PAL Robotics.}}

%\cvitem{\listitemsymbol}{"DmpBbo" 
%\href{https://github.com/stulp/dmpbbo}{\faGithub \underline {}} 
%\href{https://www.youtube.com/watch?v=jkaRO8J_1XI}{\faPlayCircle \underline{}}
%\href{https://www.youtube.com/watch?v=MAiw3Ke7bh8}{\faPlayCircle \underline{}}
%\href{https://www.youtube.com/watch?list=PLFxFrY0V2V0XbA0yW-MwFHLsK5mHOQ1mh&v=R7LWkh1UMII}{\faPlayCircle \underline{}}
%\newline{}
%\footnotesize{C++ framework for motion primitives and black-box optimization.}}

%\cvitem{\listitemsymbol}{"mekabot" 
%\href{https://github.com/graiola/mekabot}{\faGithub \underline{}} 
%\newline{}
%\footnotesize{Meka robot packages.}}

%\cvitem{\listitemsymbol}{"m3ros-control"
%\href{https://github.com/graiola/m3ros_control}{\faGithub \underline{}}
%\newline{}
%\footnotesize{C++ bridge to integrate the control layer of the Meka robot into a ROS environment.}}

%\cvitem{\listitemsymbol}{"virtual-fixtures" 
%\href{https://github.com/graiola/virtual-fixtures}{\faGithub \underline{}}
%\href{https://www.youtube.com/watch?v=dYPGD8K_-hc}{\faPlayCircle \underline{}} 
%\href{https://www.youtube.com/watch?v=K8xCxh6U_yg}{\faPlayCircle \underline{}}
%\newline{}
%\footnotesize{Library of virtual guides for co-manipulation.}}

\section{Publications}

\subsection{Journals}

\cvitem{2020}{\textbf{G. Raiola}, E. Mingo Hoffman, M. Focchi, N. Tsagarakis, C. Semini. \newline{}
"A simple yet effective whole-body locomotion framework for quadruped robots".
\newline
\emph{Frontiers in Robotics and AI.}}

\cvitem{2019}{R. Orsolino, M. Focchi, S. Caron, \textbf{G. Raiola}, V. Barasuol, C. Semini. \newline{}
"Feasible Region: an Actuation-Aware Extension of the Support Region".
\href{https://arxiv.org/abs/1903.07999}{\faFile \underline {}} 
\newline
\emph{IEEE Transactions on Robotics.}}

\cvitem{2019}{F. Stulp, \textbf{G. Raiola}. \newline{}
"DmpBbo: A versatile Python/C++ library for Function Approximation, Dynamical Movement Primitives, and Black-Box Optimization".
\href{https://joss.theoj.org/papers/10.21105/joss.01225}{\faFile \underline {}} 
\newline
\emph{The Journal of Open Source Software.}}

\cvitem{2018}{Susana Sánchez Restrepo, \textbf{Gennaro Raiola}, Joris Guerry, Evelyn D'Elia, Xavier Lamy and Daniel Sidobre. \newline{}
"Towards an Intuitive and Iterative 6D Virtual Guides Programming Framework for Human-Robot Comanipulation".
\href{https://www.researchgate.net/publication/337937262_Towards_an_Intuitive_and_Iterative_6D_Virtual_Guide_Programming_Framework_for_Assisted_Human-Robot_Comanipulation}{\faFile \underline {}} 
\newline
\emph{Robotica.}}

\cvitem{2017}{\textbf{Gennaro Raiola}, Carlos Cardenas Alberto, Tadele Shiferaw Tadele, Theo De Vries, Stefano Stramigioli. \newline{}
"Development of a Safety and Energy Aware Impedance Controller for Collaborative Robots".
\href{https://hal.archives-ouvertes.fr/hal-01884291/document}{\faFile \underline {}} 
\newline
In \emph{IEEE Robotics and Automation Letters.} 
\newline
\small{The contents of this paper were also selected by ICRA'18 Program Committee for presentation at the Conference.}}

%\vspace{0.25cm}

\cvitem{2017}{S. Chitta, E. Marder-Eppstein, W. Meeussen, V. Pradeep, A. Rodriguez Tsouroukdissian, J. Bohren, D. Coleman, B. Magyar, \textbf{G. Raiola}, M. Ludtke and E. Perdomo Fernandez. \newline{}
"ros\_control: A generic and simple control framework for ROS".
\href{https://hal.archives-ouvertes.fr/hal-01662418/document}{\faFile \underline {}} 
\newline
\emph{The Journal of Open Source Software.}}

%\vspace{0.25cm}

\cvitem{2017}{\textbf{Gennaro Raiola}, Susana Sanchez Restrepo, Pauline Chevalier, et al. \newline{}
"Co-manipulation with a Library of Virtual Guiding Fixtures".
\href{https://hal.archives-ouvertes.fr/hal-01663467/document}{\faFile \underline {}} 
\newline
\emph{Autonomous Robots, Special Issue on Learning for Human-Robot Collaboration.}}

%\vspace{0.25cm}

\subsection{Conferences}

\cvitem{2017}{Pauline Chevalier, \textbf{Gennaro Raiola}, Brice Isableu, Jean-Claude Martin, Christophe Bazile and Adriana Tapus. \newline{}
"Do Sensory Preferences of Children with Autism Impact an Imitation Task with a Robot?".
%\href{https://hal-cea.archives-ouvertes.fr/hal-01170974/document}{\faFile \underline {}} 
\newline
\emph{Conference on Human-Robot Interaction (HRI).}}

%\vspace{0.25cm}

\cvitem{2017}{Susana Sanchez Restrepo, \textbf{Gennaro Raiola}, Pauline Chevalier, Xavier Lamy, and Daniel Sidobre. \newline{}
"Iterative Virtual Guides Programming for Human-Robot Comanipulation".
\href{https://hal.archives-ouvertes.fr/hal-01763775/document}{\faFile \underline {}} 
\newline
\emph{IEEE International Conference on Advanced Intelligent Mechatronics (AIM).}}

%\vspace{0.25cm}

\cvitem{2015}{\textbf{Gennaro Raiola}, Xavier Lamy, and Freek Stulp. \newline{}
"Co-manipulation with Multiple Probabilistic Virtual Guides".
\href{https://hal-cea.archives-ouvertes.fr/hal-01170974/document}{\faFile \underline {}} \newline
\emph{International Conference on Intelligent Robots and Systems (IROS).}}

%\vspace{0.25cm}

\cvitem{2015}{\textbf{Gennaro Raiola}, Pedro Rodriguez-Ayerbe, Xavier Lamy, Sami Tliba, and Freek Stulp. \newline{}
"Parallel Guiding Virtual Fixtures: Control and Stability".
\href{https://hal-cea.archives-ouvertes.fr/hal-01250101/document}{\faFile \underline {}} \newline
\emph{IEEE Multi-Conference on Systems and Control (MSC).}}

%\vspace{0.25cm}

\cvitem{2014}{Freek Stulp, Laura Herlant, Antoine Hoarau, and \textbf{Gennaro Raiola}. \newline
"Simultaneous On-line Discovery and Improvement of Robotic Skill".
\href{https://www.researchgate.net/publication/281886645_Simultaneous_On-line_Discovery_and_Improvement_of_Robotic_Skill_Options}{\faFile \underline {}} 
\newline
\emph{International Conference on Intelligent Robots and Systems (IROS).}}

%\vspace{0.25cm}

\cvitem{2013}{Freek Stulp, \textbf{Gennaro Raiola}, Antoine Hoarau, Serena Ivaldi, and Olivier Sigaud. \newline{}
"Learning Compact Parameterized Skills with a Single Regression".
\href{https://www.researchgate.net/publication/280852232_Learning_Compact_Parameterized_Skills_with_a_Single_Regressiong}{\faFile \underline {}} 
\newline
\emph{IEEE-RAS International Conference on Humanoid Robots.}} 

%\section{Service}

%\cvitem{Open-source}{Maintainer of ROS packages.}

%\cvitem{Research}{Reviewer for international conferences and journals:
%\begin{itemize}
%\item Autonomous Robots (Springer).
%\item IEEE Robotics and Automation Letters (RA-L).
%\item The International Conference on Robotics and Automation (ICRA).
%\item International Conference on Intelligent Robots and Systems (IROS).
%\end{itemize}
%}

%\vspace{0.25cm}

\section{Languages}
\cvitem{italian}{native proficiency}
\cvitem{english}{professional working proficiency}
\cvitem{french}{limited working proficiency}

%\vspace{0.25cm}


%\section{References}
%\cvitem{Dr. Freek Stulp}
%{Head of Department \href{mailto:freek.stulp@dlr.de}{\faEnvelope} 
%\newline{}
%Department of Cognitive Robotics
%Institute of Robotics and Mechatronics
%DLR - German Aerospace Center (Previously ENSTA-ParisTech)}
%\vspace{0.1cm}
%\cvitem{Dr. Xavier Lamy}{Research Engineer \href{mailto:xavier.lamy@cea.fr}{\faEnvelope}  
%\newline{}
%Laboratory of interactive robotics at CEA-LIST}
%\vspace{0.1cm}
%\cvitem{Dr. Adolfo Rodriguez Tsouroukdissian}{Senior Robotics Engineer \href{mailto:adolfo.rodriguez@pal-robotics.com}{\faEnvelope}  
%\newline{} 
%Intermodalics (Previously PAL Robotics)}


%Autorizzo il trattamento dei miei dati personali ai sensi del Decreto Legislativo 30 giugno 2003, n. 196 “Codice in materia di protezione dei dati personali”

%----------------------------------------------------------------------------------------

\end{document}
