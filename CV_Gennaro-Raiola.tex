%%%%%%%%%%%%%%%%%%%%%%%%%%%%%%%%%%%%%%%%%
% "ModernCV" CV and Cover Letter
% LaTeX Template
% Version 1.1 (9/12/12)
%
% This template has been downloaded from:
% http://www.LaTeXTemplates.com
%
% Original author:
% Xavier Danaux (xdanaux@gmail.com)
%
% License:
% CC BY-NC-SA 3.0 (http://creativecommons.org/licenses/by-nc-sa/3.0/)
%
% Important note:
% This template requires the moderncv.cls and .sty files to be in the same 
% directory as this .tex file. These files provide the resume style and themes 
% used for structuring the document.
%
%%%%%%%%%%%%%%%%%%%%%%%%%%%%%%%%%%%%%%%%%

%----------------------------------------------------------------------------------------
%	PACKAGES AND OTHER DOCUMENT CONFIGURATIONS
%----------------------------------------------------------------------------------------

\documentclass[12pt,a4paper,sans]{moderncv} % Font sizes: 10, 11, or 12; paper sizes: a4paper, letterpaper, a5paper, legalpaper, executivepaper or landscape; font families: sans or roman

\moderncvstyle{classic} % CV theme - options include: 'casual' (default), 'classic', 'oldstyle' and 'banking'
\moderncvcolor{blue} % CV color - options include: 'blue' (default), 'orange', 'green', 'red', 'purple', 'grey' and 'black'

\usepackage{savesym}
\savesymbol{fax}
%
%\usepackage{bbding,pifont}
%\usepackage[ps]{skak}
%\usepackage{china2e}

\usepackage{marvosym}
\usepackage{fontawesome}
\usepackage{lipsum} % Used for inserting dummy 'Lorem ipsum' text into the template
\usepackage[utf8]{inputenc}
\usepackage[scale=0.75]{geometry} % Reduce document margins
%\setlength{\hintscolumnwidth}{3cm} % Uncomment to change the width of the dates column
%\setlength{\makecvtitlenamewidth}{10cm} % For the 'classic' style, uncomment to adjust the width of the space allocated to your name

%----------------------------------------------------------------------------------------
%	NAME AND CONTACT INFORMATION SECTION
%----------------------------------------------------------------------------------------

\firstname{Gennaro} % Your first name
\familyname{Raiola} % Your last name

% All information in this block is optional, comment out any lines you don't need
\title{Curriculum Vitae}
\address{Via Ariosto 3}{16159 Genova, IT}
\email{gennaro.raiola@gmail.com}
\homepage{https://github.com/graiola}
%{web} % The first argument is the url for the clickable link, the second argument is the url displayed in the template - this allows special characters to be displayed such as the tilde in this example
%\extrainfo{additional information}
%\photo[100pt][0.4pt]{photo_sea.png} % The first bracket is the picture height, the second is the thickness of the frame around the picture (0pt for no frame)
%\quote{}

%----------------------------------------------------------------------------------------

\begin{document}

\makecvtitle % Print the CV title



%\begin{center}
%\textit{Hard-working, flexible Robotics Engineer with a strong profession in clean code development; seeking new challenges to foster autonomous devices with powerful hardware and artificial intelligence; Passionate and goal oriented team player \\
%\#Robotics, \#Control, \#Machine Learning, \#Software Engineering, \#ROS, \#C++, \#Python, \#Matlab,  
%}
%\end{center}


\section{Fields of interest}
Robotics, computer science, motion planning, human-robot interaction, machine learning, controls, DevOps, hacking.

\section{Profession}
\cvitem{09/2017-current}{\emph{Post-Doc} @ Istituto Italiano di Tecnologia (IIT), Genoa, Italy \Mundus \href{https://www.iit.it/research/lines/dynamic-legged-systems}{ \underline{www}} – Development and maintenance of robot real-time control frameworks and communication systems with EtherCAT. Software architecture development with the integration of control modules in current frameworks to make the robots capable of executing tasks in complex environments. Open source code adaptation, ROS and ROS packages integration on the robots. DevOps  processes: create and maintain fully automated CI/CD pipelines for code deployment using GitLab-CI, apt servers and Docker containers. Development of software for safety to protect robot hardware and operators. Development and integration of GUI to interact with the robot. Cooperation with external work groups (mainly Moog, Vodafone and Inail) in the development of robot software.}

\cvitem{03/2017-07/2017}{\emph{Post-Doc} @ at Robotics and Mechatronics group, University of Twente, Enschede, The Netherlands. \Mundus \href{https://www.ram.ewi.utwente.nl/}{ \underline{www}} - Development of a Safety and Energy aware impedance controller for the KUKA LWR 4+. Supervision of students in the laboratories.}

\cvitem{01/2016-12/2016}{\emph{Ph.D. student in Robotics} @ CEA-List (French Alternative Energies and Atomic Energy Commission - Laboratory for Integration of Systems and Technology), Gif-sur-Yvette, France. \Mundus \href{http://www-list.cea.fr/en/}{ \underline{www}} - Transfering research results from the PhD to a Startup at CEA-List (\href{https://www.isybot.com/}{ISybot}). Development of a force controller to generate virtual guides through kinesthetic teaching. Integration of the controller in the software framework of the startup's collaborative robot.}

\cvitem{05/2013-12/2013}{\emph{Research Engineer in Motion Control of Humanoid Robots} @ PAL Robotics S.L., Barcelona, Spain. \Mundus \href{http://www.pal-robotics.com}{ \underline{www}} - Development, testing and design of ROS-Control and ROS-Controllers to implement a Hardware Abstraction Layer for different kinds of robots (humanoids, manipulators and mobile robots). Implementation through ROS-Control of an inverse kinematics solver with task optimization for REEM-H and REEM-C robots.}

\cvitem{09/2012-02/2013}{\emph{Internship} @ ENSTA-ParisTech and UPMC-ISIR, Paris, France. \Mundus \href{http://www.isir.upmc.fr/}{ \underline{www}} - Development of a library in Matlab and C++ to generate Motion Primitives and perform Skills Optimization for humanoid robots (MEKA, NAO, ICub and Pepper). Maintenance of MEKA robot libraries.}

%----------------------------------------------------------------------------------------
%	EDUCATION SECTION
%----------------------------------------------------------------------------------------
\section{Education}
%\cvitem{2017-current}{\emph{Post-Doc} \newline{} at ADVR-DLS, Istituto Italiano di Tecnologia (IIT), Genova, Italy. \newline \Mundus \href{https://dls.iit.it/}{ \underline{www}}.}
\cvitem{2014-2016}{\emph{Ph.D. student in Robotics} @ Université Paris-Saclay, Palaiseau, France. \Mundus \href{https://www.universite-paris-saclay.fr/fr}{ \underline{www}}.}
%\cvitem{2016-2017}{\emph{Ph.D. student in Robotics} \newline{} at CEA (French Alternative Energies and Atomic Energy Commission), Gif-sur-Yvette, France. \Mundus \href{http://www-list.cea.fr/en/}{ \underline{www}}.}
%\cvitem{2014-2016}{\emph{Ph.D. student in Robotics} \newline{} at INRIA - Flowers / ENSTA-ParisTech, Palaiseau, France. \Mundus \href{http://cogrob.ensta-paristech.fr/}{ \underline{www}}.}
%\cvitem{2012}{\emph{Master's thesis work on Motion Primitives} under the supervision of Prof. Freek Stulp and Prof. Bruno Siciliano.}
\cvitem{2009-2012}{\emph{Master's Degree (M.Sc) with honor in Automation and Control Engineering} \newline{} given by the University of Naples "Federico II", Naples, Italy.}
\cvitem{2006-2009}{\emph{Bachelor's Degree (B.Sc) in Computer Engineering} \newline{} given by the University of Naples "Federico II", Naples, Italy.}

%\vspace{0.25cm}

%\section{Research Interests}

%Controls, Software Engineering, Machine Learning, Motion Primitives, Collaborative robotics. 

%\section{Research}
%\cvitem{Post-doc}{Energy based control strategies for collaborative robots \newline{}
%	\footnotesize{
%	\underline{Keywords}: Safety in robotics, Impedance Control, Collaborative robotics.}}

%\cvitem{Ph.D. Thesis}{"Co-manipulation with a library of Virtual Guides" \newline{}
%	\footnotesize{\underline{Supervisors}: Prof. Freek Stulp (ENSTA-ParisTech), Dr. Xavier Lamy (CEA-List). \newline{}
%	\underline{Keywords}: Collaborative robotics, Industrial applications, Controls, Machine Learning.}}

%\cvitem{M.Sc. Thesis}{"Learning Parameterized Skills through Models with Expanded Kernels" \newline{} 
%	\footnotesize{\underline{Supervisors}: Prof. Bruno Siciliano (PRISMA lab.) and Prof. Freek Stulp (ENSTA-ParisTech). \newline{}
%	\underline{Keywords}: Motion Primitives, Machine Learning.}}

%\vspace{2.0cm}

%\hspace{2em}

%\section{Technical Skills and Software projects}

%\subsection{Technical Skills}
%\cvitem{Programming}{C/C++, Python, MATLAB, ROS, Orocos, Qt, Boost, RTAI/Xenomai, Eigen, php}
%\cvitem{Systems/Tools}{Linux/Ubuntu, CMake, Latex, QtCreator, Gazebo, Eclipse, git, VirtualBox, Docker, EtherCAT}
%\cvitem{Hardware}{Arduino}

%\subsection{Software projects}

\section{Technical skills}

\cvitem{\listitemsymbol}{Proficient in the following programming languages: C, C++ and Matlab}
\cvitem{\listitemsymbol}{Competent with Python and Bash scripting.}
\cvitem{\listitemsymbol}{Competent with Qt, Eigen, ROS and Boost libraries.}
\cvitem{\listitemsymbol}{Excellent knowledge of GIT.}
\cvitem{\listitemsymbol}{Excellent knowledge of CMake and Makefile for managing the build process of software and Doxygen for code documentation.}
\cvitem{\listitemsymbol}{Competent with Docker and Virtual Machines deployment for testing and development.}
\cvitem{\listitemsymbol}{Deep knowledge of Linux-based operating systems (Ubuntu, Kali, Debian).}
\cvitem{\listitemsymbol}{Experienced with real time operating systems (RTAI Linux, Xenomai Linux, RT-PREEMPT) and Kernel configuration.}
\cvitem{\listitemsymbol}{Good understanding of UML process.}
\cvitem{\listitemsymbol}{Good understanding of Agile Scrum process.}

\section{Selected open-source software projects}

\cvitem{\listitemsymbol}{"ros-control" 
\href{https://github.com/ros-controls}{\faGithub \underline{}}
\newline{}
\footnotesize{Ros packages to make controllers generic to all robots.}}

\cvitem{\listitemsymbol}{"Stack-of-Task" 
\href{http://stack-of-tasks.github.io/authors.html}{\faGithub \underline{}}
\href{https://www.youtube.com/watch?v=lmE5RmvkAhQ}{\faPlayCircle \underline{}} 
\newline{}
\footnotesize{Integration of the stack of tasks inverse kinematics solver on the REEM-H and REEM-C robots at PAL Robotics.}}

\cvitem{\listitemsymbol}{"DmpBbo" 
\href{https://github.com/stulp/dmpbbo}{\faGithub \underline {}} 
\href{https://www.youtube.com/watch?v=jkaRO8J_1XI}{\faPlayCircle \underline{}}
\href{https://www.youtube.com/watch?v=MAiw3Ke7bh8}{\faPlayCircle \underline{}}
\href{https://www.youtube.com/watch?list=PLFxFrY0V2V0XbA0yW-MwFHLsK5mHOQ1mh&v=R7LWkh1UMII}{\faPlayCircle \underline{}}
\newline{}
\footnotesize{C++ framework for motion primitives and black-box optimization.}}

\cvitem{\listitemsymbol}{"mekabot" 
\href{https://github.com/graiola/mekabot}{\faGithub \underline{}} 
\newline{}
\footnotesize{Meka robot packages.}}

\cvitem{\listitemsymbol}{"m3ros-control"
\href{https://github.com/graiola/m3ros_control}{\faGithub \underline{}}
\newline{}
\footnotesize{C++ bridge to integrate the control layer of the Meka robot into a ROS environment.}}

\cvitem{\listitemsymbol}{"virtual-fixtures" 
\href{https://github.com/graiola/virtual-fixtures}{\faGithub \underline{}}
\href{https://www.youtube.com/watch?v=dYPGD8K_-hc}{\faPlayCircle \underline{}} 
\href{https://www.youtube.com/watch?v=K8xCxh6U_yg}{\faPlayCircle \underline{}}
\newline{}
\footnotesize{Library of virtual guides for co-manipulation.}}

\section{Publications}

\subsection{Journals}

\cvitem{2018}{Susana Sánchez Restrepo, \textbf{Gennaro Raiola}, Joris Guerry, Evelyn D'Elia, Xavier Lamy and Daniel Sidobre. \newline{}
"Towards an Intuitive and Iterative 6D Virtual Guides Programming Framework for Human-Robot Comanipulation".
\newline
\emph{Under review at Robotica} }

\cvitem{2017}{\textbf{Gennaro Raiola}, Carlos Cardenas Alberto, Tadele Shiferaw Tadele, Theo De Vries, Stefano Stramigioli. \newline{}
"Development of a Safety and Energy Aware Impedance Controller for Collaborative Robots".
\href{https://hal.archives-ouvertes.fr/hal-01884291/document}{\faFile \underline {}} 
\newline
In \emph{IEEE Robotics and Automation Letters.} 
\newline
\small{The contents of this paper were also selected by ICRA'18 Program Committee for presentation at the Conference.}}

%\vspace{0.25cm}

\cvitem{2017}{S. Chitta, E. Marder-Eppstein, W. Meeussen, V. Pradeep, A. Rodriguez Tsouroukdissian, J. Bohren, D. Coleman, B. Magyar, \textbf{G. Raiola}, M. Ludtke and E. Perdomo Fernandez. \newline{}
"ros\_control: A generic and simple control framework for ROS".
\href{https://hal.archives-ouvertes.fr/hal-01662418/document}{\faFile \underline {}} 
\newline
In \emph{The Journal of Open Source Software.}}

%\vspace{0.25cm}

\cvitem{2017}{\textbf{Gennaro Raiola}, Susana Sanchez Restrepo, Pauline Chevalier, et al. \newline{}
"Co-manipulation with a Library of Virtual Guiding Fixtures".
\href{https://hal.archives-ouvertes.fr/hal-01663467/document}{\faFile \underline {}} 
\newline
In \emph{Autonomous Robots, Special Issue on Learning for Human-Robot Collaboration.}}

%\vspace{0.25cm}

\subsection{Conferences}

\cvitem{2017}{Pauline Chevalier, \textbf{Gennaro Raiola}, Brice Isableu, Jean-Claude Martin, Christophe Bazile and Adriana Tapus. \newline{}
"Do Sensory Preferences of Children with Autism Impact an Imitation Task with a Robot?".
%\href{https://hal-cea.archives-ouvertes.fr/hal-01170974/document}{\faFile \underline {}} 
\newline
In \emph{Conference on Human-Robot Interaction (HRI).}}

%\vspace{0.25cm}

\cvitem{2017}{Susana Sanchez Restrepo, \textbf{Gennaro Raiola}, Pauline Chevalier, Xavier Lamy, and Daniel Sidobre. \newline{}
"Iterative Virtual Guides Programming for Human-Robot Comanipulation".
\href{https://hal.archives-ouvertes.fr/hal-01763775/document}{\faFile \underline {}} 
\newline
In \emph{IEEE International Conference on Advanced Intelligent Mechatronics (AIM).}}

%\vspace{0.25cm}

\cvitem{2015}{\textbf{Gennaro Raiola}, Xavier Lamy, and Freek Stulp. \newline{}
"Co-manipulation with Multiple Probabilistic Virtual Guides".
\href{https://hal-cea.archives-ouvertes.fr/hal-01170974/document}{\faFile \underline {}} \newline
In \emph{International Conference on Intelligent Robots and Systems (IROS).}}

%\vspace{0.25cm}

\cvitem{2015}{\textbf{Gennaro Raiola}, Pedro Rodriguez-Ayerbe, Xavier Lamy, Sami Tliba, and Freek Stulp. \newline{}
"Parallel Guiding Virtual Fixtures: Control and Stability".
\href{https://hal-cea.archives-ouvertes.fr/hal-01250101/document}{\faFile \underline {}} \newline
In \emph{IEEE Multi-Conference on Systems and Control (MSC).}}

%\vspace{0.25cm}

\cvitem{2014}{Freek Stulp, Laura Herlant, Antoine Hoarau, and \textbf{Gennaro Raiola}. \newline
"Simultaneous On-line Discovery and Improvement of Robotic Skill".
\href{https://www.researchgate.net/publication/281886645_Simultaneous_On-line_Discovery_and_Improvement_of_Robotic_Skill_Options}{\faFile \underline {}} 
\newline
In \emph{International Conference on Intelligent Robots and Systems (IROS).}}

%\vspace{0.25cm}

\cvitem{2013}{Freek Stulp, \textbf{Gennaro Raiola}, Antoine Hoarau, Serena Ivaldi, and Olivier Sigaud. \newline{}
"Learning Compact Parameterized Skills with a Single Regression".
\href{https://www.researchgate.net/publication/280852232_Learning_Compact_Parameterized_Skills_with_a_Single_Regressiong}{\faFile \underline {}} 
\newline
In \emph{IEEE-RAS International Conference on Humanoid Robots.}} 

\section{Service}

\cvitem{Open-source}{Maintainer of ROS packages.}

\cvitem{Research}{Reviewer for international conferences and journals:
\begin{itemize}
\item Autonomous Robots (Springer).
\item IEEE Robotics and Automation Letters (RA-L).
\item The International Conference on Robotics and Automation (ICRA).
\item International Conference on Intelligent Robots and Systems (IROS).
\end{itemize}
}

\section{Languages}
\cvitem{italian}{native proficiency}
\cvitem{english}{professional working proficiency}
\cvitem{french}{limited working proficiency}
\cvitem{spanish}{basic knowledge}

%\vspace{0.25cm}


%\section{References}
%\cvitem{Dr. Freek Stulp}
%{Head of Department \href{mailto:freek.stulp@dlr.de}{\faEnvelope} 
%\newline{}
%Department of Cognitive Robotics
%Institute of Robotics and Mechatronics
%DLR - German Aerospace Center (Previously ENSTA-ParisTech)}
%\vspace{0.1cm}
%\cvitem{Dr. Xavier Lamy}{Research Engineer \href{mailto:xavier.lamy@cea.fr}{\faEnvelope}  
%\newline{}
%Laboratory of interactive robotics at CEA-LIST}
%\vspace{0.1cm}
%\cvitem{Dr. Adolfo Rodriguez Tsouroukdissian}{Senior Robotics Engineer \href{mailto:adolfo.rodriguez@pal-robotics.com}{\faEnvelope}  
%\newline{} 
%Intermodalics (Previously PAL Robotics)}


%Autorizzo il trattamento dei miei dati personali ai sensi del Decreto Legislativo 30 giugno 2003, n. 196 “Codice in materia di protezione dei dati personali”

%----------------------------------------------------------------------------------------

\end{document}
