%%%%%%%%%%%%%%%%%%%%%%%%%%%%%%%%%%%%%%%%%
% "ModernCV" CV and Cover Letter
% LaTeX Template
% Version 1.1 (9/12/12)
%
% This template has been downloaded from:
% http://www.LaTeXTemplates.com
%
% Original author:
% Xavier Danaux (xdanaux@gmail.com)
%
% License:
% CC BY-NC-SA 3.0 (http://creativecommons.org/licenses/by-nc-sa/3.0/)
%
% Important note:
% This template requires the moderncv.cls and .sty files to be in the same 
% directory as this .tex file. These files provide the resume style and themes 
% used for structuring the document.
%
%%%%%%%%%%%%%%%%%%%%%%%%%%%%%%%%%%%%%%%%%

%----------------------------------------------------------------------------------------
%	PACKAGES AND OTHER DOCUMENT CONFIGURATIONS
%----------------------------------------------------------------------------------------

\documentclass[12pt,a4paper,sans]{moderncv} % Font sizes: 10, 11, or 12; paper sizes: a4paper, letterpaper, a5paper, legalpaper, executivepaper or landscape; font families: sans or roman

\moderncvstyle{casual} % CV theme - options include: 'casual' (default), 'classic', 'oldstyle' and 'banking'
\moderncvcolor{blue} % CV color - options include: 'blue' (default), 'orange', 'green', 'red', 'purple', 'grey' and 'black'

\usepackage{savesym}
\savesymbol{fax}
%
%\usepackage{bbding,pifont}
%\usepackage[ps]{skak}
%\usepackage{china2e}

\usepackage{marvosym}
\usepackage{fontawesome}
\usepackage{lipsum} % Used for inserting dummy 'Lorem ipsum' text into the template
\usepackage[utf8]{inputenc}
\usepackage[scale=0.75]{geometry} % Reduce document margins
%\setlength{\hintscolumnwidth}{3cm} % Uncomment to change the width of the dates column
%\setlength{\makecvtitlenamewidth}{10cm} % For the 'classic' style, uncomment to adjust the width of the space allocated to your name

%----------------------------------------------------------------------------------------
%	NAME AND CONTACT INFORMATION SECTION
%----------------------------------------------------------------------------------------

\firstname{Gennaro} % Your first name
\familyname{Raiola} % Your last name

% All information in this block is optional, comment out any lines you don't need
\title{Curriculum Vitae}
\address{78, Bld des Marechaux}{91120 Palaiseau, FR}
\email{gennaro.raiola@gmail.com}
\homepage{perso.ensta-paristech.fr/\textasciitilde raiola/}
%{web} % The first argument is the url for the clickable link, the second argument is the url displayed in the template - this allows special characters to be displayed such as the tilde in this example
%\extrainfo{additional information}
%\photo[100pt][0.4pt]{photo_sea.png} % The first bracket is the picture height, the second is the thickness of the frame around the picture (0pt for no frame)
%\quote{}

%----------------------------------------------------------------------------------------

\begin{document}

\makecvtitle % Print the CV title



\begin{center}
%\textit{Hard-working, flexible Robotics Engineer with a strong profession in clean code development; seeking new challenges to foster autonomous devices with powerful hardware and artificial intelligence; Passionate and goal oriented team player \\
%\#Robotics, \#Control, \#Machine Learning, \#Software Engineering, \#ROS, \#C++, \#Python, \#Matlab,  
%}
\end{center}

%----------------------------------------------------------------------------------------
%	EDUCATION SECTION
%----------------------------------------------------------------------------------------
\section{Education}
\cvitem{2016-current}{\emph{Ph.D. student in Robotics} \newline{} at CEA-List (French Alternative Energies and Atomic Energy Commission) \newline \Mundus \href{http://www-list.cea.fr/en/}{ \underline{www}}.}
\cvitem{2014-2015}{\emph{Ph.D. student in Robotics} \newline{} at Robotics and Computer Vision Lab - ENSTA-ParisTech \Mundus \href{http://cogrob.ensta-paristech.fr/}{ \underline{www}}.}
%\cvitem{2012}{\emph{Master's thesis work on Motion Primitives} under the supervision of Prof. Freek Stulp and Prof. Bruno Siciliano.}
\cvitem{2010-2012}{\emph{Master's Degree (M.Sc) with honor in Automation and Control Engineering} \newline{} given by the University of Naples "Federico II".}
\cvitem{2006-2010}{\emph{Bachelor's Degree (B.Sc) in Computer Engineering} \newline{} given by the University of Naples "Federico II".}

\vspace{0.25cm}

\section{Profession}
%\subsection{Employer}

\cvitem{2012-2013}{\emph{Research Engineer in Motion Control of Humanoid Robots} \newline{} 
		at PAL Robotics S.L. Barcelona \Mundus \href{http://www.pal-robotics.com}{ \underline{www}}.} %%{{\Mundus \underline {www}}}}

\vspace{0.25cm}

\section{Research}

\cvitem{Ph.D. Thesis}{"Probabilistic virtual guides for co-manipulation" \newline{}
	\footnotesize{\underline{Supervisors}: Prof. Freek Stulp (ENSTA-ParisTech), Dr. Xavier Lamy (CEA-List). \newline{}
	\underline{Topics}: Cobotics, Industrial applications, Controls, Machine Learning.}}

\cvitem{M.Sc. Thesis}{"Learning Parameterized Skills through Models with Expanded Kernels" \newline{} 
	\footnotesize{\underline{Supervisors}: Prof. Bruno Siciliano (PRISMA lab.) and Prof. Freek Stulp (ENSTA-ParisTech). \newline{}
	\underline{Topics}: Motion Primitives, Machine Learning.}}

\vspace{2.0cm}

\hspace{2em}

\section{Technical Skills and Software projects}

\subsection{Technical Skills}
\cvitem{Programming}{C/C++, Python, ROS, MATLAB, php, Qt, Gazebo, Boost, RTAI, Eigen}
\cvitem{Systems/Tools}{Linux/Ubuntu, CMake, Latex, QtCreator, Eclipse, git, svn}
\cvitem{Hardware}{Arduino}

\subsection{Software projects}

\cvitem{Software}{"virtual-fixtures" 
\href{https://github.com/graiola/virtual-fixtures}{\faGithub \underline{}} 
\href{https://www.youtube.com/watch?v=K8xCxh6U_yg}{\faPlayCircle \underline{}}
\newline{}
\footnotesize{Library of virtual guides for co-manipulation.}}

\cvitem{Software}{"mekabot" 
\href{https://github.com/graiola/mekabot}{\faGithub \underline{}} 
\newline{}
\footnotesize{Meka robot packages.}}

\cvitem{Software}{"m3ros-control"
\href{https://github.com/graiola/m3ros_control}{\faGithub \underline{}}
\newline{}
\footnotesize{C++ bridge to integrate the control layer of the Meka robot into a ROS environment.}}

\cvitem{Software}{"DmpBbo" 
\href{https://github.com/stulp/dmpbbo}{\faGithub \underline {}} 
\href{https://www.youtube.com/watch?v=MAiw3Ke7bh8}{\faPlayCircle \underline{}}
\href{https://www.youtube.com/watch?list=PLFxFrY0V2V0XbA0yW-MwFHLsK5mHOQ1mh&v=R7LWkh1UMII}{\faPlayCircle \underline{}}
\newline{}
\footnotesize{C++ framework for motion primitives and black-box optimization.}}

\cvitem{Software}{"Stack-of-Task" 
\href{https://github.com/stack-of-tasks}{\faGithub \underline{}}
\href{https://www.youtube.com/watch?v=lmE5RmvkAhQ}{\faPlayCircle \underline{}} 
\newline{}
\footnotesize{Integration of the stack of tasks inverse kinematics solver on the Reem-H and Reem-C robots at PAL Robotics.}}


\section{Publications}

\cvitem{2017}{Pauline Chevalier, Gennaro Raiola, Brice Isableu, Jean-Claude Martin, Christophe Bazile and Adriana Tapus. \newline{}
"Do Sensory Preferences of Children with Autism Impact an Imitation Task with a Robot?".
%\href{https://hal-cea.archives-ouvertes.fr/hal-01170974/document}{\faFile \underline {}} 
\newline
Under Review for \emph{Conference on Human-Robot Interaction (HRI).}}

\cvitem{2017}{Susana Sanchez Restrepo, Gennaro Raiola, Pauline Chevalier, Xavier Lamy, and Daniel Sidobre. \newline{}
"Iterative Virtual Guides Programming for Comanipulation Robots".
%\href{https://hal-cea.archives-ouvertes.fr/hal-01170974/document}{\faFile \underline {}} 
\newline
Under Review for \emph{The International Conference on Robotics and Automation (ICRA).}}

\cvitem{2015}{Gennaro Raiola, Xavier Lamy, and Freek Stulp. \newline{}
"Co-manipulation with Multiple Probabilistic Virtual Guides".
\href{https://hal-cea.archives-ouvertes.fr/hal-01170974/document}{\faFile \underline {}} \newline
In \emph{International Conference on Intelligent Robots and Systems (IROS).}}

\vspace{0.25cm}

\cvitem{2015}{Gennaro Raiola, Pedro Rodriguez-Ayerbe, Xavier Lamy, Sami Tliba, and Freek Stulp. \newline{}
"Parallel Guiding Virtual Fixtures: Control and Stability".
\href{https://hal-cea.archives-ouvertes.fr/hal-01250101/document}{\faFile \underline {}} \newline
In \emph{IEEE Multi-Conference on Systems and Control (MSC).}}

\vspace{0.25cm}

\cvitem{2014}{Freek Stulp, Laura Herlant, Antoine Hoarau, and Gennaro Raiola. \newline
"Simultaneous On-line Discovery and Improvement of Robotic Skill".
\href{http://perso.ensta-paristech.fr/~raiola/publications/stulp14simultaneous.pdf}{\faFile \underline {}} \newline
In \emph{International Conference on Intelligent Robots and Systems (IROS).}}

\vspace{0.25cm}

\cvitem{2013}{Freek Stulp, Gennaro Raiola, Antoine Hoarau, Serena Ivaldi, and Olivier Sigaud. \newline{}
"Learning Compact Parameterized Skills with a Single Regression".
\href{http://perso.ensta-paristech.fr/~raiola/publications/stulp13learning.pdf}{\faFile \underline {}} \newline
In \emph{IEEE-RAS International Conference on Humanoid Robots.}} 

\vspace{0.25cm}

\section{Service}

\cvitem{}{Reviewer for international conferences:
\begin{itemize}
\item IEEE-RAS International Conference on Humanoid Robots.
\item International Conference on Intelligent Robots and Systems (IROS).
\end{itemize}
}


\section{Languages}
\cvitem{italian}{native proficiency}
\cvitem{english}{professional working proficiency}
\cvitem{french}{limited working proficiency}
\cvitem{spanish}{basic knowledge}

\vspace{0.25cm}

\section{References}
\cvitem{Dr. Freek Stulp}
{Head of Department \href{mailto:freek.stulp@dlr.de}{\faEnvelope} 
\newline{}
Department of Cognitive Robotics
Institute of Robotics and Mechatronics
DLR - German Aerospace Center (Previously ENSTA-ParisTech)}
\vspace{0.1cm}
\cvitem{Dr. Xavier Lamy}{Research Engineer \href{mailto:xavier.lamy@cea.fr}{\faEnvelope}  
\newline{}
Laboratory of interactive robotics at CEA-LIST}
\vspace{0.1cm}
\cvitem{Dr. Adolfo Rodriguez Tsouroukdissian}{Senior Robotics Engineer \href{mailto:adolfo.rodriguez@pal-robotics.com}{\faEnvelope}  
\newline{} 
Intermodalics (Previously PAL Robotics)}


%Autorizzo il trattamento dei miei dati personali ai sensi del Decreto Legislativo 30 giugno 2003, n. 196 “Codice in materia di protezione dei dati personali”

%----------------------------------------------------------------------------------------

\end{document}
